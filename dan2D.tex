\section{Optimizing dividends and \ci \cite{Gaj,AGLW} \la{s:AG}}
We refer to \cite{Gaj,AGLW}  for the formulation of this stochastic control problem.

 In this section we revisit the   problem of optimizing numerically the value of "bounded buffer $(-a,0,b)$ policies", which
consist in allowing capital injections smaller than a given $a$ and declaring bankruptcy at the first time when the size of the overshoot below 0 exceeds $a,$ and  pay dividends when the reserve reaches an upper barrier $b$.
\subsection{The cost function and its optimization \la{s:cost}}
We  present here a synthesis of the  results of \cite{Gaj,AGLW} (in order  to relate the results, one needs to replace  $\g$ in the  objective of \cite{Gaj} by  $1/k$).  They are
expressed  in terms of the functions
\be \la{RG} \bc R_a(x) =\l \int_0^x W_q(x-y) \;   \ovl F(y+a)  \; dy \\G_a(x)=\l \int_0^x  W_q(x-y) \;  m_y(a) \; \; dy, \; \; \; \; m_y(a)=\int_0^a z f(y+z)dz\\S_a(x)=Z_q(x)+  R_a(x)\ec \ee
(\cite{Gaj} use $s_c,r_c$, instead of $G_a(x),R_a(x)$, \resp).

\beR The relation \eqr{J0} below earns the function $S(x,a)$ the name of scale function for our problem, which basically means that it will appear in several different problems involving a reflecting barrier at $b$ and a limited reflection buffer $(-a,0]$. See \cite[Rem. 8]{AGLW} for some other examples, and see the equation \cite[(3)]{Gaj} for an additional example. \eeR

\beR \la{r:idG} Cf. \cite[Lem. A.4]{Gaj}, these functions \saty
\be \la{idG} G_a(x)+a R_a(x)=\int_0^a R_y(x)dy =\l \int_0^x W_q(x-y) \int_0^a \ovl F(y+z) dz dy \Lra \bc  G_{a'}(x)= -a  R_{a'}(x)\\G_a'(x)+a R_a'(x)= \int_0^a R_y'(x)dy \ec,\ee
where
 $R_{a'}(x),G_{a'}(x)$ denote derivatives with respect to the subscript, and
where we added a missing minus in the last statement of \cite[Lem. A.4]{Gaj}.
\eeR
\beXa
In the particular case of \expoj, the functions \eqr{RG}  become \cite{AGLW}
$$\bc G_a(x)=  C_\q(x) m(a), \; R_a(x)=
C_\q(x)e^{- \mu  a}, \\ C_q(x)=c W_q(x)- Z_q(x), \; m(a)=\int_0^a y \; F( \md y)=\frac{1-e^{-\mu a } (\mu a +1)}{\mu }\ec,$$
as follows
from the identities $\ovl F(y+a)=e^{-\mu a} \ovl F(y), m_y(a)= e^{-\mu y} m(a)$.

Note also the formulas \be \la{heur} \bc
R_a(x) =  C_\q(x) (1-F(a))\\  G_a(x)=
C_\q(x) \int_0^a y \; F( \md y) %{+k \si W_q(x), \text{ cf sols 2,3?}}
\ec, \ee
which will be used below as a heuristic approximation in non-\expo\ cases.

\eeXa

\beXa

Consider now the more general case when the claims are distributed according to a matrix exponential density generated by a row vector $\vb$ and by an invertible matrix $B$ of order $n$, which are such that the vector $\vb e^{ x B}$ is decreasing componentwise to $0$, and $\vb . \vo \neq 0 $, with $\vo %, \bff b
$ a  column vector. As customary, we  restrict \wlo to the case
when  $\vb$ is a probability vector, and  $\vb . \bff 1 =1$, so that  $$\ovl F(x)=%(\vb . \vo)^{-1}
\vb e^{ x B} \vo$$  is a valid survival function.


 The matrix versions of our  functions are:

  \be \la{mpf} \bc R_a(x) =\l \int_0^x W_q(x-y) \;   \ovl F(y+a)  \; dy = \l \vb \int_0^x W_q(x-y) \;e^{ y B}     \; dy  \; e^{ a B} \vo= \vec C_q(x) e^{ a B} \vo
 \\ m_y(a)=\int_0^a z f(y+z)dz=\vb  \; e^{ y B} \; \int_0^a z \;e^{ z B} (-B)    \; dz \;\vo=\vb   \; e^{ y B} M(a) \vo
 \\G_a(x)=\l \int_0^x  W_q(x-y) \;  m_y(a) \; \; dy= \vec C_q(x) M(a) \vo \; \; \; \; \ec, \ee
 where
\be \bc C_q(x)=\l  \int_0^x W_q(x-y) \;e^{ y B}     \; dy\\
\vec C_q(x)=\l \vb \int_0^x W_q(x-y) \;e^{ y B}     \; dy \ec. \ee

The product  formulas \eqr{mpf} may also be established directly in the \PH case,  using the conditional independence of the ruin probability of the overshoot size -see section \ref{s:me}.
\eeXa

\beP \cite[Thm. 4]{Gaj} {\bf Cost function  for $(a,b)$ policies} \la{l:intfper}

 For a % {\per}
\CL process  (\cP) with \expoj, let
$$J_x=J^{a,b}(x):=
\mathbb{E}_x\pp{\int_0^{\ta}e^{-qt}\pr{\md \D_t -k \;\md\C_t}}  $$
denote the \eddc\ associated to policies consisting in paying capital injections  with proportional cost $k\geq 1$, provided that the severity of ruin is smaller than $a>0$, and paying dividends as soon as the  process reaches some upper level $b$.


 Then,\BEN \im
 \be J_x= \bc k G_a(x) +J_0^{(a,b)} S_a(x)=k G_a(x) +\fr{1-k G_a'(b)}{S_a'(b)} S_a(x),  &x \in[0, b]\\k x+J_0^{(a,b)}
&x \in[-a, 0]\\ 0 & x \leq -a \ec. \la{struct}\ee
\im  For fixed  $b\geq 0$, the optimality equation $\fr{\partial}{\partial a} J_0^{a,b}=0$  \mbw \be \la{paa}
 k a= J_0^{a,b} \Eq J_{-a}^{a,b}= 0. \ee

\EEN
\eeP
\prf The first statement is \cite[Thm. 4]{Gaj}, and the second is a consequence of \eqr{idG}.\qed

In the \expo\ case, further simplification is possible. In particular, we will take advantage of properties of the Lambert-W function, which were not exploited in \cite{AGLW}.
\beC {\bf Cost function and  optimality conditions in the \expo\ case}
\BEN \im
  \be J_0^{(a,b)}=
\fr{1-k \; m(a) C_\q'(b)}{(1-F(a)) C_\q'(b) + q W_q(b)}=\frac{{\g(b)}-k \;  m(a) }
{ 1-F(a)+q \th(b)}, \la{J0}
 \ee
where we put $$ \g(b)=\fr{1}{C_q'(b)}, \th(b)=\fr{W_q(b)}{C_\q'(b)}.$$

 \im For fixed  $a\geq 0$, the optimality equation $\fr{\partial}{\partial b} J_0^{a,b}=0$  \mbw

\be  J_0^{a,b}  =j(b), \; j(b):= \fr{ \g'(b)}{q  \th'(b)}. \la{jb}\ee

At a critical point $(a^*,b^*), a^*>0, b^* >0$,  \wmh\ $J_0^{a^*,b^*}  =j(b^*)=k a^* \Lra$
\be    a^* =s(b^*), s(b):=\fr{j(b)}k. \la{J0a}\ee

In conclusion, $b^*$ for such critical points may be computed solving
\be \la{str} %q \th(b) j(b)+ \fr k{\mu} \pr{1- e^{-\mu a}}=\g(b) \Eq
 \eta(b):=\fr{\g(b)}{\th(b)}-q  j(b) - \fr k{\mu \th(b)}  F\pr{\fr{j(b) }{k}}=0. \ee


\beR  a) The important equation \eqr{J0a} identifies  the optimal buffer associated with a dividends barrier $b$ via the explicit function $s(b)$. In the general framework of \cite{Gaj}, $s(b)$ is only defined implicitly as solution of   \cite[(6)]{Gaj}.

b) For fixed  $b\geq 0$, the optimality equation $$\fr{\partial}{\partial a} J_0^{a,b}=0  \Eq  J_0^{(a,b)}= k a  =
 {
 \fr{\g(b)- k m(a)}
 {e^{-\mu a}+ q\th(b)}}$$
 \mbw also as
 $k a=\fr{\g(b)- k \fr{1- e^{-\mu a}}\mu}{q\th(b)}.$

\eeR



\im  In the special case $b^*=0$, the optimality equation \eqref{paa} implies that $a^*=a_k:=a_{k,0}$ satisfies
the simpler equation
\begin{align}\label{StructureEqa}
\d(k,a):=c -k\pr{a q + \fr{\l}{\mu} F( a)  }=0. % \; c =c +
\end{align}

\EEN






 Rewriting the equation \eqr{StructureEqa} as $z e^z =\fr \l q e^{ g}, z=\mu a +g$) implies that the solution  is
$$\mu a =-g + L_W\pr{\fr \l q e^{ g}}, g=\fr{\l }{ q}-\fr{\mu c}{k q},$$
where $L_W$ denotes the real Lambert-W function \cite{corless1996lambertw,boyd1998global,brito2008euler,vazquez2019psem}
    (this observation is missing in \cite{AGLW}).
\eeC

\prf 2. From \eqr{J0}, the optimality equation $\fr{\partial}{\partial b} J_0^{a,b}
=0$ simplifies to
$$J_0^{(a,b)}=\frac{{\g'(b)} }
{ q \th'(b)}=j(b)(=-\fr{C_\q''(b)}{q\pr{ W_q'(b) C_\q'(b)-C_\q''(b)  W_q(b)}}), $$ and recalling $J_0^{(a,b)}=k a$ yields the result.
\qed

\beR Without switching to $\g, \th$, the previous computation is
more complicated
\bea J_0^{(a,b)}=
\fr{-k \; m(a) C_\q''(b)}{(1-F(a)) C_\q''(b) + q W_q'(b)}\eea

\eeR


%\iffalse
We have now a further look at  the functions introduced in Proposition \ref{l:intfper}.
\Itm $\g$ is increasing-decreasing (from $\fr c \l$ to $0$), with a maximum at
the root of $C_q''(x)=0$, which is
\be \la{bb}\bar{b}:=\frac{1}{\Phi_q-\rho_-}\log\pr{\frac{\rho_-^2}{\Phi_q^2}},\ee and $\th$ increases from %$0$ to $1/\Fq$.
%(and hence $\th$ increases from
$\th(0)=\fr{1/c}{\l/c}=\fr 1{\l}$ to $\th(\I)=\fr 1{c \Fq - q }$.%=\fr{\mu \Fq+1}{\l}$ .

The following result is to be found, albeit with somewhat different notations, in \cite[Proof of Theorem 11, A2]{AGLW}.


\beL
The function $j(b)=\frac{\g'(b)}{q \th'(b)}$ is decreasing, with  $j(0)=\fr{\l}{\mu q}\fr{-C_\q''(0)}{(C_\q'(0))^2}=
\fr{c \mu -  \pr{q +\l}}{\mu q}$. If  $c \mu -  \pr{q +\l} >0$, then $\bar{b}>0$ defined in \eqr{bb} is the unique positive root  of $j({b}) $.
\eeL



\beR Introducing
$$\eta(b,a):=\fr {\g(b)}{\th(b)} -k \pr{ q a+ \fr 1{\mu \th(b)} F\pr{a}}=\fr {1}{W_q(b)} -k \pr{ q a+ \fr 1{\mu \th(b)} F\pr{a}},$$ we note  by using $C_q'(0)=\fr \l c, W_q(0)=\fr 1 c$ that $$\eta(0,a):=c -k \pr{ q a+ \fr {\l} {\mu }  F\pr{a}}=\d(k,a).$$

The continuous function $\pp{0,\bar{b}}\ni b\mapsto\eta(b),$
\be \la{et0} \eta(b):=\eta(b,s(b))=\fr {1}{W_q(b)}   -
  q j(b) -\fr {k} {\mu \th(b)}  F\pr{s(b)}, \; s(b):=\fr{j(b) }k,\ee
  already defined in \eqr{str},
  will play an important role below.

   Note that \bea && \eta(0)=\fr c \l- \fr 1 \l \pr{c - \fr {q +\l}\mu}-\fr {k}\mu F\pr{\frac{j(0)}{k }}=\fr {1}{\l \mu} \pr{\l+  {q}-\l {k} F\pr{\frac{j(0)}{k }}}=\fr {1}{\l }\delta(k)<0.%, \forall k>k^*,
\eea

 and    $\eta\pr{\bar{b}} =\fr{1}{W_q\pr{\bar{b}}}>0$. \Thr $\eta(b)=0$  has at least one solution of in $[0,\bar{b}]$; the first such solution will be denoted by $b^*$.

 % Note that the function $s(b)$, which figures also in \cite[(6)]{Gaj}  as $c(x)$, is explicit in our setup.
\eeR

\beR Putting $h=\fr {1}{q \th(b)}-\fr {\mu } k \fr {\g(b)}{q \th(b)} ,$ the structure equation \eqr{str} may be rewritten as:
\bea (\fr {\mu } k j(b)+ h) e^{\fr {\mu } k j(b)+ h}=\fr {e^{h }}{q \th(b)}  \Lra \fr {\mu } k j(b)+h = L_w(\fr {e^{h }}{q \th(b)} ).\eea
%\be     \eta(b):=\fr {1}{W_q(b)}\pr{1 - k\fr {C_q'(b)}{\mu } F\pr{s(b)}}   -k  q s(b)+ =0. \la{str}\ee
\eeR

The following (new!) result relates the
dichotomy domains  to the  Lambert-W function.

\beL
The  function \be \d(k):=\d(k,j(0)/k)=\d(k,\fr{c \mu -  \pr{q +\l}}{k \mu q})=\mu^{-1}
\pr{\lambda+q-\lambda k\pr{1-e^{-\frac{c\mu-\lambda-q}{qk}}}}
\ee
has a unique  \nne\ root
$k^*$  iff
\begin{equation}
\label{Cheapk}c \mu>\lambda^{-1} \pr{\lambda+q}^2>\pr{\lambda+q }.\end{equation}
%and $P > P_0$.,

Explicitly,
\be \la{ks} k^*=\fr{q + \l} \l \fr{f}{f + L_W\pr{- f e^{-f}}}, \; f =\fr{ \l}{q+\l} \fr{c\mu-\pr{\lambda+
q }}q. \ee




\eeL
\prf  The equation to be solved is similar to \eqr{StructureEqa}, but this time the unknown is $k$.

Putting $d =\fr{c\mu-\pr{\lambda+
q }}q $ %and $z=\frac{ d}{k}$,
 reduces the equation $\d(k)=0$ to
\bea k e^{-\frac{d}{k}}=  k-\fr{q+\l}\lambda  \Eq  e^{-\frac{d}{k}}=  1 -\fr{q+\l}{\l k} \Eq 1= e^{\frac{d}{k}}
\pr {1 -\fr{(q+\l)d}{d \l k}}:=e^{z} \pr {1 -z/f}, f=\fr{d \l}{q+\l}.
\eea

Rewriting the latter as $-f=e^{z} \pr {z -f}$  we recognize, by putting $z=y+f$,  an equation
reducible   to $ y e^{y} =- f e^{-f}, $ whose  real solution is $$y =L_W\pr{- f e^{-f}},$$ where $L_W$ denotes the real Lambert-W function. The final solution
is \eqr{ks}.
