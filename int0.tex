
\section{Introduction}

We recall first  the Cram\'{e}r-Lundberg risk
model  with an added Brownian perturbation \cite{dufresne1991risk,AA}
\begin{align} \la{CLp}
X_t = x + c t + \s B(t) -S_t, S_t=\sum_{i=1}^{N_\lambda (t)} C_i.
\end{align}
Here $x \geq 0$ is the initial surplus, $c \geq 0$ is the linear premium rate. The
 $C_i$'s, $i=1,2,...$ are independent identically distributed (i.i.d)
random variables with distribution ${F(z)=F_C(z)}$ representing
nonnegative jumps arriving after independent exponentially distributed
times with mean $1/\lambda$, and $N_\lambda (t)$ denotes the associated Poisson
process counting the arrivals of claims on the interval $[0,t]$. Finally, $\s B(t),  \s >0$
is an independent Brownian perturbation.
 
 {\bf First passage theory}  concerns the first passage times above and below, and the hitting  time of a level $b$. For any process $X_t, t \geq 0$, these are defined by
\begin{equation}\la{fpt}
\begin{aligned}
\tb &= \tb^{X}=\inf\{t\geq 0: X_t> b\},\\
\ta&=\ta^X=\inf\{t\geq 0: X_t< a\},\\
T_{\{b\}}&=T_{\{b\}}^X =\inf\{t\geq 0: X_t=b\},
\end{aligned}
\end{equation}
with $\inf\emptyset=+\infty$. The upper script $X$ will be typically  omitted, as well as the signs $+,-$, when they are clear from the context.


Ruin happens when, for the first time, a jump takes $X_t$ below 0. Risk theory revolved initially around evaluating and minimizing the probability of ruin. As more and more tools were developed, and the need arose, insurance companies became interested in maximizing aspects of company value related to ruin. This lead to the study of optimal dividend policies.  As suggested by De Finetti in the 1950's \cite{deF} -- see also \cite{miller1961dividend} -- an interesting objective is that of maximizing the expected value of the sum of discounted future dividend payments until the time of ruin.

%\sec{De Finetti optimization of dividends  \la{s:div}}

The most important class of dividend policies is that of a constant barrier
at $b$, which modifies the surplus only when $X_t>b$, by a lump payment
bringing the surplus at $b$, and then keep it there by Skorokhod
reflection, until the next negative jump.  In financial terms, in the
absence of a Brownian component, this amounts to paying out all the income
while at $b$.  In case of Brownian perturbation, Skorokhod reflection means
keeping the process above the barrier by minimal capital injections
(whenever necessary), or below a barrier, by taking out dividends (if
necessary) \cite{sk62}.

In presence of the barrier at $b$, the de Finetti objective (the expected
value of the sum of discounted future dividend payments until ruin) has a
simple expression \cite{APP} in terms of the so called ``scale function''
$W$ introduced by \cite{Suprun,Ber}:
\begin{equation}
V^{b]}(x)=E^{b]}_x\le[\int_{[0,T_0^{b]}]}e^{-\q
   t}d \D_t^b \ri]= \bc \frac{W_\q( x)}{W_{\q}^ {\prime}(b)}, & x \leq
   b\\x-b + \frac{W_\q( b)}{W_{\q}^ {\prime}(b)}, & x>b\ec, \la{div}
\end{equation}
where $T_0^{b]}$ is the time of ruin, $q$ denotes the discount rate,
%$\D_t^b:=\le(\sup_{0 \leq s\leq t} X_{s}-b\ri)_+$
$\D_t^b$
the total local time at
$b$ before time $t$, and $E^{b]}$ the law of the process reflected from
above at $b$ and absorbed at $0$ and below.

\begin{comment}
\fn[4]{Formula \eqr{WLT}
reflects the representation $$V^{b]}(x)=E[ e^{-q \tb} ; \tb
< \tz] Eazb_b\le[\int_{[0,T_0^{b]}]}e^{-\q t}d \D_t\ri]=E[ e^{-q \tb}
; \tb < \tz]\; Eazb_b\le[ \D_{T_0^{b]} \wedge \kil_q}\ri],$$ and the fact
that the local time $\D_t$ at $b$ with reflection at $b$ is an
exponential \rv.}
\end{comment}

The scale function 
$W_{\q}(x):\R \to [0, \infty), \q \geq 0$ is defined on the positive
half-line by the Laplace transform
\be \label{WLT}
\H W_\q (s):=\int_0^\infty  \mathrm{e}^{-s x}  W_{\q}(x) d x = \frac {1} {\k(s)-\q} , \quad \for s > \Fq,
 \ee
where the ``symbol" $\k(s)$ (also called \cgf) is defined in \eqr{kf} in
Section~\ref{s:RT} where we provide the necessary background information,
and $\Fq$ is the unique nonnegative root of the \CL equation
\be \Fq:= \sup \{ s \geq 0: \k(s)
- \q= 0\}, \quad \q \geq 0. \label{Fq} \ee

The scale function $ W_{\q}(x)$ is continuous and increasing on
$[0,\I)$ \cite{Bingham}, \cite[Thm.~VII.8]{Ber}, \cite[Thm.~8.1]{Kyp}. It
may have \how many inflection points (such an example is depicted in
Figure \ref{f:pd}), and these play an important role in the optimization
of \divs \ \cite{APP,Schmidli,APP15}.  For convenience, $ W_{\q}(x)$ is
extended to be $0$ on $\R_-$. An important fact that will be exploited is
that the \LT \ of our function has a unique non-negative pole $\Fq$,
see \eqr{WLT} and \eqr{Fq}.


This paper aims at computing/approximating the scale function $W_{\q}(x)$,
using its moments. The techniques being used are classic: \Pd and Laguerre
expansions.  The order $(m,n)$ Pad\'e approximation of a function $g(x)$ is
a rational function in the form
\[
R(x)=\frac{a_0+a_1 x+a_2 x^2+...+a_m x^m}{1+b_1 x+b_2 x^2+...+b_n x^n}
\]
for which
$R(0)=g(0),R'(0)=g'(0),R''(0)=g''(0),...,R^{(m+n)}(0)=g^{(m+n)}(0)$.  In
the context of probability distributions, given a density function $f(x)$
and its \LT \ $\H f(s)$, the inverse Laplace transform of the order $(m,m)$
Pad\'e approximant of $\H f(s)$ provides a matrix exponential approximation
of $f(x)$ that matches the first $2m$ moments of $f(x)$ (including $m_0$).
In \cite{avram2011moments} this approach was used to approximate ruin probabilities.  In
this paper we develop the same approach to approximate the scale function
$W_{\q}(x)$ (Section~\ref{s:Pade}).  An extension of the above idea is the
so-called two-point Pad\'e approximation which allows to match not only the
moments of $W_{\q}(x)$ but also the behavior of the function at 0, i.e., to
match $W_{\q}(0)$, $W'_{\q}(0),...$ (Section~\ref{sec:low1}).
For more details of this extension see  \cite{ABH} where ruin
probabilities are approximated.


\begin{comment}
  \beR   Note that when $\q=0$, the scale function $W_0$ coincides, up to
  a \prop  \ct,  with the well-studied  \surp \ (just compare \eqr{WLT}
  with the famous \PK  \LT \ \eqr{PK0}).
  It appears thus that at the slight additional expense of inverting
  the  \LT \ \eqr{WLT} instead of  \eqr{PK0}, one may obtain the solution of numerous sophisticated control problems \cite{Kyp,AGV} (which are however similar
  in effort to obtaining ruin probabilities).
\eeR
\end{comment}

Let us draw attention now to several numeric challenges which were absent in the \rp \prob.
\BEN \im
{\bf Optimizing dividends} starts by optimizing the so called ``barrier function'' \be H_D(b):=\frac {1}{ W'_\q(b)}, \; b \geq 0, \la{GDeF}\ee
   and   the optimal dividend policy is  often simply a barrier strategy at its   maximum. This is the case in particular when the barrier function $H_D(b)$  is differentiable
with
\be H_D'(0) >0 Eq W''_\q(0) <0\ee
and has a unique local maximum $b^*>0 \Lra W''_\q(b^*)=0$; then this $b^*$ yields the optimal dividend policy, and
the optimal barrier function,
  \be \la{dbpr} V(x):= sup_{b \geq 0} V^{b]}(x)=V^{b^*]}(x),\ee
  turns out to be the largest concave minorant  of  $W_\q(x)$.\fn[4]{Even when barrier strategies do not  achieve the  optimum,  and  multi-band policies must be used instead, constructing the solution must start by
  determining the global maximum of the barrier function   \cite{AM05,Schmidli,APP15}.}

\im {\bf The challenge of multiple inflection points}.
  In the presence of several inflection points, however,  the optimal policy is multiband \cite{AM05,Schmidli,Loef,APP15}.
The first numerical examples of multiband policies  were produced in \cite{AM05,Loef}, with  Erlang claims $Erl_{2,1}$. However, it was shown in \cite{Loef} that multibands cannot occur when $W'_\q(x)$ is increasing after its  last global minimum  $b^*$ (i.e., when no local minima are allowed after the global minimum).


  \cite{Loef} further made the interesting  observation that for Erlang claims $ER_{2,1}$  (which are non-monotone), multiband policies may occur  for volatility \pars \ $\s$ smaller than a threshold value, but  barrier policies (with non-concave value function!) will occur when $\s$ is large  enough.

 Figure \ref{f:pd} displays the first derivative $W_\q'(x),$ for $\s^2/2 \in \{\frac{1}{2}, 1,\frac{3}{2},2\}$. The last two values yield barrier policies with non-concave value function, due to the presence of an inflection point in the interior of the interval $[0,b^*]$.
   \figu{pd-eps-converted-to}
   {Graphs of the Loeffen example for $\k(s)= \frac{\s^2 s^2}{2}+c \; s+\l
   \left(\frac{1}{(s+1)^2}-1\right), c=\frac{107 }{5}, \l=10, q=\frac{1}{10} $, $\s^2/2 \in \{ \color{blue} 1/2, \color{yellow}1, \color{green} 3/2, \color{red} 2 \color{black}\} $.}{0.6}




\EEN

 Below we will investigate whether our approximations are precise enough to
 yield reasonable approximations for $ W''_\q(0) $ and the root(s) of $
 W''_\q(\cdot)$.

{\bf Special features}. While our methods consist essentially of \Pd \
and \LTW \LT \ inversion, we found that exploiting the special features of
our problem is useful. These are:\BEN \im including known values of
$W_\q(0), W_\q'(0)$ (using thus two-point \Pds). \im shifting the
approximations around $\Fq$ specified in \eqr{Fq}, which transforms
$W_\q(x)$ into a \surp. As a consequence, we end up using a certain
judicious choice of the Laguerre exponential decay parameter \eqr{al},
which is usually left to be tuned by the user in the \LTW \
method \cite{weideman1999algorithms}.
\EEN

