\section{Which exponential   approximation? \la{s:DV}}
\ssec{Some de Vylder-type approximations for the \rp\ \la{s:DVr} }
In the simplest case of \expoj\ and $\s=0$, the  formula for the \rp\ is
\begin{equation} \la{Ruiex}
\Rui(x) = P_x[ \exists t \geq 0: X_t < 0]= \frac{1}{1+  {\th}} \ exp \left( - \frac{x  {\th} m_1^{-1}}{1+  {\th}} \right),
\end{equation}
where $\th=\fr{c -\l m_1}{\l m_1}$ is the \lc.
This formula may serve  as the {\bf simplest approximation} for \procs\ with general claims, simply by plugging the mean of the claims.


{\bf The Renyi and De Vylder's approximations}.
More sophisticated is the Renyi exponential
approximation
\begin{equation*}
\Rui_R(x) =  \frac{1}{1+  {\th}} \ exp \left( - \frac{x  {\th} m_R^{-1}}{1+  {\th}} \right), m_R=\frac{m_{2}}{2m_{1}};
\end{equation*}

This formula can  be obtained as a \tPd\   of the \LT, which conserves also the value $\Rui(0)=(1+\th)^{-1}$ \cite{AP14}. It may be also viewed heuristically as an exponential
approximation of the {\bf excess/severity density} $$f_e(x)=\fr{\ovl F(x)}{m_1}=\fr{1- F(x)}{m_1}.$$ Heuristically, it makes more sense to make $f_e(x)$ exponential instead of the original density $f(x)$, since $f_e(x)$ is a monotone function (and also an important component of the \PK formula for the \LT\ $\H \Rui(s)=\int_0^\I e^{- s x} F( dx)$) -- see \cite{ramsay1992practical,AP14}.




More moments may be put to work in the \deV
\begin{equation}
\Rui_{DV}(x) = \frac{1}{1+ \T {\th}} \ exp \left( - \frac{x \T {\th} \T m^{-1}}{1+ \T {\th}} \right), \; \T {m}=\frac{m_{3}}{3m_{2}}=\H m_{3}, \;  \T \th =\frac{2 m_{1}m_{3}}{3m_{2}^{2}}\th=\frac{\H m_{3}}{\H m_{2}}\th.
\end{equation}
Interestingly, the result may be expressed in terms of the so-called "normalized moments" \be {\H m_i}=\fr{m_i}{i \; m_{i-1}}\ee  introduced in \cite{bobbio2005matching}.

The \deV\ parameters above may be obtained either from  \BEN \im a \Pd\ of the \LT\ of the \rp\ \cite{avram2011moments}, or \im by equating the first  cumulants of our process to those of a \proc\ with  exponentially distributed claim sizes of mean $\T m$,  and  modified $\l, c$  \cite{de1978practical};  (however   $p=c -\l m_1$ must be conserved, since this is the first cumulant). This results -- see \eqr{ap3} -- in the modified \paras
$$\T m =\H m_3, \;  \T {\lambda}
= \frac{9 m_{2}^{3}}{2 m_{3}^{2}}\lambda=\fr{ m_2/2} { \H m_3^2} \l, \; \;  \T c= c -\l m_1+ \T \l \T m= c -\l m_1+  \l \frac{ m_{2}^{2}}{2 m_{3}}$$
where we gave also expressions involving the less standard normalized moments $\H m_k$ of \cite{bobbio2005matching}.
\EEN

The first derivation via Pad\'e shows that higher order approximations may be easily obtained as well (but they might not be admissible, due to negative values).

The second derivation of  De Vylder is a {\bf process approximation} (i.e., independent of the problem considered); as such, it may  be  applied to other functionals of interest  besides \rps ($W_q(x)$, dividend barriers, etc), simply by plugging the modified \para s in the exact formula for the \rp\ of the simpler process.






\iffalse
\section{Can we minimize the \rp, by proportional \rei? \la{s:DVr} }

Let us consider first the mean premium principle, with $\th_R \geq \th$ and retention level (potentially dependent on $x$). As \wk, we may forget about the reinsurer, and say instead that we modify the premium to
and the \lc\ to
\be \la{Tc} \T c=\l m_1 \pr{1+ \th - (1-a) \pr{ \th_R +1}}=\l m_1 \pr{1+ \T \th},\ee
\be \la{pos} \T \th=\th - (1-a) \pr{ \th_R +1},\ee
with $a \geq \fr {1 +\th_R-\th}{1+ \th_R } $, to ensure positivity in \eqr{pos}.
 \Fr we must also multiply the mean by $a$. %the level of the naive approximation

 The \rp\ has one critical point with $a$ satisfying  a quadratic equation (but, taking $a$ at the critical point  yields negative $\T \th$). A careful analysis reveals that the optimum is at $a=1$?.

 This continues to be the case for the mean-variance premium principle (Shihao?).
 The optimization becomes non-trivial for excess of loss \rei?.
\fi
