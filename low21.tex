



\ssec{Pad\'e approximations of the scale functions $W_q, Z_q, q >0$ \la{s:DVW}}




Our goal in this section is to investigate whether the approximations described previously
 are precise enough to yield reasonable estimates for quantities important in control like $ W''_\q(0) $ and the global minimum   of $ W'_\q(\cdot)$, which yields (typically) the \deF\ optimal dividends barrier $b_{DeF}$.  Note that other \perf\ measures like the dual optimal dividends barrier, and the reflected optimal dividends barrier could be investigated as well. 

 Most of our approximations may be obtained by plugging appropriate values  in  the exact formula \eqr{Wexp} for $W_q$, for  the \CLp\ with \expoc. We will call them {\bf process approximations}, and we have encountered already three:
 \BEN \im   the naive, \im the Renyi, and
  \im the  de Vylder \expo\ approximating processes. \EEN
     They yield each an \app, simply  by plugging in the exact exponential formula \eqr{Wex} with $\s=0$ the appropriate  \paras.
 \beXa
  {\bf The Cram\'{e}r-Lundberg model with exponential jumps \la{s:exp}}
Consider  the Cram\'{e}r-Lundberg model
 with exponential jump sizes with mean $1/\mu$, jump
rate $\lambda$, premium rate $c>0$,
and Laplace exponent
$\k(s)=s \le(c-\fr{\lambda}{\mu+s}\ri)$, assuming $\k'(0) = c- \fr{\lambda} {\mu} > 0$. Let $\g=\mu - \l/c$ denote the adjustment coefficient, and let $\r=\fr \l{c \mu}$. Solving $\k(s)-\q=0 \Eq c s^2 + s(c \mu -\l -\q) - \q \mu=0$ for $s$ yields two distinct solutions $\g_{2} \leq 0 \leq \g_{1}=\Fq$ given by
\begin{align*}
\g_{1} =& \fr{1}{2c} \left(- \left(\mu c -\lambda - \q\right) + \sqrt{\left(\mu c -\lambda - \q \right)^2 + 4\mu \q c} \right),\\
\g_{2} =& \fr{1}{2c} \left(- \left(\mu c -\lambda - \q\right) - \sqrt{\left(\mu c -\lambda - \q \right)^2 + 4\mu \q c} \right).
\end{align*}


The $W$ scale function is:
\be \la{Wexp}  W_{\q}(x) = \fr{A_1 e^{\g_{1}x} - A_2 e^{\g_{2}x}}{c(\g_{1}-\g_{2})}  \Eq \H W_{\q}(s) = \fr{s+ \mu}{c s^2 + s(c \mu -\l -\q) - \q \mu},\ee
where $A_{1} = \mu + \g_{1}, A_{2} = \mu + \g_{2}$.

Furthermore, it is \wkt    the function $ W_{\q}'(x)$ is in this case
unimodal  with global minimum at
\be \la{ob} b_{DeF} = \frac{1}{\g_{1} - \g_{2}}
\begin{cases}\log
\frac{(\g_{2})^2 A_{2}}{(\g_{1})^2 A_{1}}=\log
\frac{(\g_{2})^2(\mu +\g_{2})}{(\g_{1})^2(\mu +\g_{1})} \quad &\text{if $ W_{\q}''(0) <  0 \Eq (\q+\lambda)^2$}-
c\lambda\mu < 0\\ 0 & \text{if $ W_{\q}''(0)\geq 0 \Eq
(\q+\lambda)^2- c\lambda\mu \geq 0$}\end{cases}, \ee since
$ W_{\q}''(0) =
\fr{(\g_{1})^2(\mu +\g_{1})-(\g_{2})^2(\mu +\g_{2})}{c(\g_{1}-\g_{2})}
=\fr{ (\q+\lambda)^2- c\lambda\mu}{c^3}$ and that the optimal strategy for the \deF \prob \ is
 the barrier strategy at level $b_{DeF}$ \cite{APP}.
\iffalse
 Also, the optimal barriers in the presence of a final penalty $P$, and of reflection with proportional costs $k$, \saty \resp
 \be \bc P  \q \Delta_\q^{(W)}(b_P)=-W_\q''(b_P)\Eq P  \q \mu \Tl c^{-2}
     e^{(\Tl + \Tq -\mu) b_P}=-W_\q''(b_P) \\k \Delta_\q^{(ZW)}(b_k)=W_\q'(b_k) \Eq k \Tl c^{-2}
     e^{(\Tl + \Tq -\mu) b_k}=W_\q'(b_k)
  \ec. \ee
\fi
\eeXa








We may  also attempt to use  \Pds, or  two-point Pad\'e approximations  of the \LT\ of $W_q$,  which incorporate into the \Pd\  the following initial values (these can be
derived easily via the initial value theorem, from the \PK \LT):
\beq \la{W0} &&W_\q(0)=  \lim_{s \to \I}s \H W_\q(s)= \frac 1 {\c}, \\&&
W_\q'(0)= \lim_{s \to \I}s\left( \fr{s}{\k(s) -\q}- W_{\q} (0) \right)=   \frac {\q + \l} {\c^2}. \la{W0p}\eeq

Furthermore, when  the jump distribution has a density  $f$, \ith:\fn[6]{This equation is important in establishing the nonnegativity of the optimal dividends barrier.}
\begin{equation} \label{e:secder}
\begin{aligned}
 W_{\q}'' (0_+) &=\lim_{s \to \I} s \left( s \left(\fr{s}{\k(s) -\q}- W_{\q} (0)\right) -  W_{\q}' (0_+) \right)=    \fr 1 c \Big( (\fr {\lambda+ \q }  c )^2 - \fr {\lambda}  c f(0) \Big).
\end{aligned}
\end{equation}


Further derivatives at $0$ could be computed, but we  stop at order $2$, since  $W_\q''(0)$ already requires estimating $f_C(0)$, which is a rather delicate task starting from real data.

 We  recall below three types of two-point \Pds\ \cite[Prop. 1]{AHPS}, and particularize them to the case when the denominator degree is $n=2$ (which are studied further below). \red{Note that the first two are of the type we saw in Example 1}.

\beP \la{p:deV} \cite[Prop. 1]{AHPS} {\bf Three \red{matrix exponential} approximations for the scale function}.
\BEN \im To secure both the values of $W_\q(0)$ and $W_\q'(0)$, take into account \eqr{W0} and \eqr{W0p}, i.e. use the \Pd  $$\H W_\q(s) \sim \fr{\sum_{i=0}^{n-1} a_i s^i}{c s^n+ \sum_{i=0}^{n-1} b_i s^i}, a_{n-1}=1, b_{n-1}=c a_{n-2}-\l -\q.$$
This yields the naive approximation  $\mu \to \fr 1{m_1}$
\be \la{W2zz} \H W_\q(s) \sim
   \frac{\frac{1}{m_1}+s}{c s^2+ s \left(\frac{c}{m_1}-\lambda
   -q\right)-\frac{q}{m_1}}.\ee




   \im To ensure $W_\q(0)=\fr 1 c$, we must only impose the behavior  specified in \eqr{W0}, i.e. use the \Pd  $$\H W_\q(s) \sim \fr{\sum_{i=0}^{n-1} a_i s^i}{c s^n+ \sum_{i=0}^{n-1} b_i s^i}, a_{n-1}=1.$$
For $n=2$, this yields the Renyi approximation  $\mu \to \fr 1{ m_R}, \l -> \lambda \fr{ m_1}{m_R}$,
\be \la{W2z} \H W_\q(s) \sim\frac{\frac{2 m_1}{m_2}+s}{c s^2 +\frac{s \left(2 c m_1-2
   \lambda  m_1^2-m_2 q\right)}{m_2}-\frac{2 m_1
   q}{m_2}}=\frac{\frac 1 {m_R}+s}{c s^2 +{s \left(\fr c {m_R}-
   \lambda \fr{ m_1}{m_R}- q\right)}-\frac{
   q}{m_R}},\ee
   where  $m_R=\frac {m_2}{2 m_1}$ is the first moment of the excess density $f_e(x)$.
     This is called  DeVylder B)  method in
    \cite[(5.6-5.7)]{GSS}, and is the \wk\ result   of  fitting the first two cumulants of the risk process.  Note that it equals the \sf\ of a \proc\
    with  \expoc\ of rate $m_R$ and with    $\l$ modified to \red{ $\l_R=\lambda \fr{ m_1}{m_R}$, and   that, since $c$ is unchanged, the latter equation is equivalent to the conservation of $\rho=\fr{\l m_1}c,$ and to the conservation of $\th$.}


 \im The pure \Pd\
\beq \la{W2} && \H W_\q(s) \sim \frac{ s+\fr{3 m_2}{m_3}}{s^2 \left( c- \l m_1 + \lambda  \fr {3 m_2^2}{2
   m_3}\right)+s \left( c\fr{3 m_2}{m_3}-\fr{3 m_1 m_2}{m_3} \lambda  - q\right)-\fr{3 m_2}{m_3} q}\no\\&&=
   \frac{ s+\fr{1}{\H m_3}}{s^2 \left( c -\l m_1 + \T \l_L    \H m_2 \right)
   +s \left( c \fr{1}{\H m_3}-\Tl_L  - q\right)-\fr{1}{\H m_3} q}, \; \Tl_L=\fr{3 m_1 m_2}{m_3} \lambda=\fr{ m_1 }{\H m_3} \lambda.\eeq



   Note that the coefficient of $s^2$ in the denominator coincides with the one in the  classic \deV, but the coefficient of $s$ doesn't. Also, $\T \l_L$  is different from the classic \deV\ \para\  $\T \l$ (also called DeVylder (A)   in
    \cite[(5.2-5.4)]{GSS}).

Finally, comparing with \eqr{Wexp} we see that this \red{cannot be viewed as an "exponential process" approximation}.
    Indeed, if it were, the first two \paras\ should  be
    $$\T m= \H m_3, \Tl_L=\fr{3 m_1 m_2}{m_3} \lambda=\fr{ m_1 }{\H m_3} \lambda. $$ Now the coefficient of $s$ imposes preserving $c$, but the coefficient of $s^2$ contradicts this (unless  $\l m_1 =  \tilde{\l}_L  {\H m_2} \Eq 2 m_1 m_3 = 3 m_2^2$.



\EEN
\eeP
\beR  In the case of \expoc,  these three  approximations are exact.
 Indeed, it suffices to check that for \expoc \ all the normalized moments are equal to $\mu^{-1}$. \eeR
 We have included a derivation of the less-known ``De Vylder-Laplace" third approximation in Section \ref{s:LdV}.
