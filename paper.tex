\documentclass[slidetop,lecno,12pt,fleqn,mathserif]{article}
\usepackage[utf8]{inputenc}
\usepackage[french]{babel}
\usepackage[T1]{fontenc}
\usepackage{amsmath,amsthm}
\usepackage{amsfonts}
\usepackage{amssymb}
\usepackage{comment}
\usepackage{mathtools,color}
\setcounter{secnumdepth}{5}
\usepackage{graphicx}
\usepackage{xcolor}
\newtheorem{Cor}{Corollary}
\newtheorem{Def}{Definition}
\newtheorem{Rem}{Remark}
\newtheorem{Exe}{Exercice}
\def\beXe{\begin{Exe}} \def\eeXe{\end{Exe}}
\newtheorem{Thm}{Theorem}
\def\beT{\begin{Thm}} \def\eeT{\end{Thm}}
\def\eeD{\end{Def}} \def\beD{\begin{Def}}
\def\beXa{\begin{Exa}} \def\eeXa{\end{Exa}}
\def\beR{\begin{Rem}} \def\eeR{\end{Rem}}
\def\beL{\begin{Lem}} \def\eeL{\end{Lem}}
\newtheorem{Lem}{Lemma}
%\newtheorem{As}{Assumption}
%{\theorembodyfont{\normalfont}
\newtheorem{Exa}{Example}
%\newtheorem{ex}{Example}
\newtheorem{Pro}{Proposition}
\def\beP{\begin{Pro}} \def\eeP{\end{Pro}}
\def\beC{\begin{Cor}}
	\def\eeC{\end{Cor}}
\def\bc{\begin{cases}
	} \def\fz{f_C(0)}
	\def\ec{\end{cases}} \def\Tl{\tilde{\lambda}}
\def\Fq{\Phi_\q} \def\eqr{\eqref} \def\asy{asymptotically } \def\LTW{Laguerre-Tricomi-Weeks}
\def\sf{scale function} \def\prob{problem} \def\probs{problems } \def\pro{probability } \def\im{\item}
   \def\pros{probabilities } \def\ia{infinite activity }
\def\deF{de Finetti } \def\oedcd{optimal expected discounted cumulative dividends } \def\MAP{Markov additive processes }
\def\GS{Gerber-Shiu } \def\D{\Delta} \def\fint{finite time} \def\surp{survival probability}
\def\lev{L\'evy }  \def\mL{{\mathcal L}} \def\T{\tilde} \def\R{\mathbf{R}}
\def\for{\forall} \def\Lra{\Longrightarrow} \def\divs{dividends} \def\cgf{cumulant generating function} \def\rp{ruin probability } \def\expc{exponential claims} \def\ija{infinite jump activity} \def\pars{parameters }
\def\ol{\overline }
\def\bep{\begin{pmatrix}}
  \def\eep{\end{pmatrix}}
\def\BEN{\begin{enumerate}}  \def\BI{\begin{itemize}}
\def\EEN{\end{enumerate}}   \def\EI{\end{itemize}}
\def\BM{Brownian motion } \def\r{\rho}
\def\beq{\begin{eqnarray}}\def\eeq{\end{eqnarray}}
\def\bea{\begin{eqnarray*}}
\def\eea{\end{eqnarray*}}
\def\le{\left} \def\ri{\right}
\def\T{\widetilde}  \def\H{\widehat}
 \def\WH{Wiener-Hopf }
\def\I{\infty}  \def\a{\alpha}
 \def\b{\beta}  \def\val{\vec{\a }}
\def\g{\gamma}  \def\d{\q} \def\de{\delta}
\def\z{\zeta}  \def\th{\theta}
\def\e{\epsilon} \def\k{\kappa} \def\l{\lambda}
\def\lm{\Pi}
\def\strf{straightforward}
 \def\exp{exponential } \def\Ip{In particular, } \def\cP{compound Poisson } \def\LZ{Lokka-Zervos alternative }
\def\pros{probabilities } \def\fund{fundamental } \def\gen{generalized }
\def\rvs{random variables } \def\ovl{\overline} \def\upa{\uparrow} \def\gene{generalization } \def\alt{alternatively, } \def\mix{mixed exponential } \def\Lm{L\'evy measure}
\def\procs{processes} \def\proc{process} \def\capi{capital injections}
\def\expoc{exponential claims}  \def\expoj{exponential jumps}
\def\asy{asymptotic } \def\deV{de Vylder } \def\app{approximation}
   \def\wrb{w^{b[}} \def\hrb{f^{b[}} \newcommand{\prf}{\noindent{\bf Proof:\ }} \def\os{one-sided } \def\tbz{T^{[0}_b} \def\tba{T^{[a}_{b}}
   \def\mU{{\mathcal U}} \def\ith{it holds that } \def\mD{{\mathcal D}}
    \def\mZ{{\mathcal \zeta}}  \def\mW{{\mathcal \omega}} \def\mI{{\mathcal I}}
    \def\wkt{well-known and easy to check that } \def\Pd{Pad\'e approximation } \def\Pds{Pad\'e approximations} \def\mpr{more precisely } \def\ts{two-sided } \def\wk{well-known } \def\pros{probabilities }
\def\prop{proportionality } \def\ct{constant } \def\cts{constants } \def\para{parameter }
 \def\sub{subordinator } \def\rps{ruin probabilities }
 \def\str{strong Markov property} \def\ts{two-sided } \def\wk{well-known} \def\FT{fixed Talbot algorithm } \def\la{\label} \def\rt{ruin theory}
 \def\lm{\nu}
\def\sn{spectrally negative }
\def\sp{spectrally positive }
\def\LE{Laplace exponent } \def\LT{Laplace transform}
\def\LTs{Laplace transforms}
\def\PCL{perturbed \CL } \def\q{q}  \def\c{c}
\def\Eq{\Leftrightarrow} \def\fp{first passage }
\def\CL{Cram\'er-Lundberg }
\def\PK{Pollaczek-Khinchine } \def\How{However, }
\def\KP{Kolmogorov Pearson } \def\ith{it holds that }
\def\me{matrix exponential }
\def\fr{\frac} \def\how{however }
 \def\sub{subordinator } \def\fe{for example}
\long\def\symbolfootnote[#1]#2{
	\begingroup
	\def\thefootnote{\fnsymbol{footnote}}\footnote[#1]{#2}
	\endgroup}
\def\fn{\symbolfootnote}\def\mts{moments}
\def\t{\tau} \def\ta{T_{a,-}}  \def\tb{T_{b,+}} \def\s{\sigma}
\def\tz{T_{0}} \def\tret{T_{\{0\}}}
\def\Rui{\Psi} \def\sRui{\ovl{\Rui}} \def\ruik{\psi_\d}
 \def\bf{\bfseries}  \def\rm{\textrm} \def\it{\itshape}
\def\mG{\mathcal G} \def\tGe{\tilde{\mG}} \newcommand{\ep}{\mathbf{e}_{\q}}
\newcommand{\be}{\begin{equation}}
\newcommand{\ee}{\end{equation}}
\newcommand{\ba}{\begin{array}}
\newcommand{\ea}{\end{array}}
\newcommand{\norm}[1]{|\!|#1|\!|}

\newcommand{\figu}[3]{
\begin{figure}[!h]
\begin{center}
{\includegraphics[width=#3\textwidth]{#1}}
\end{center}
\vspace{-0.2cm}
\caption{\hspace{0.25cm}#2\label{f:#1}}
\end{figure}
}
\newcommand{\red}{\textcolor[rgb]{1.00,0.00,0.00}}
\newcommand{\blue}{\textcolor[rgb]{0.00,0.00,1.00}}
\newcommand{\green}{\textcolor[rgb]{0.00,0.50,0.50}}
\author{Rim ADENANE and Florin Avram}
\title{A \gene of De Vylder's formula and an application}
\date{\today}
%\setbeamercovered{transparent}
%\setbeamertemplate{navigation symbols}{}
%\logo{}
%\institute{}
%
%\subject{}

\begin{document}
\maketitle
\tableofcontents

\section{Abstract}
In $1978$ \textit{De Vylder} proposed a method of approximating the ultimate ruin probability in the classical  Cramèr-Lundberg risk model, by replacing it via a simpler \proc \ with \expoj.
In this paper we consider a \gene : we approximate the underlying \proc \ by a \BM perturbed \proc \ with  \expoj. We show numerically that, as expected, the obtained approximation works typically better, and illustrate this with  important applications including the optimization of dividends and the computation of \rps under reinsurance.

%\input{int}

\section{Introduction}

We recall first  the Cram\'{e}r-Lundberg risk
model  with an added Brownian perturbation \cite{dufresne1991risk,AA}
\begin{align} \la{CLp}
X_t = x + c t + \s B(t) -S_t, S_t=\sum_{i=1}^{N_\lambda (t)} C_i.
\end{align}
Here $x \geq 0$ is the initial surplus, $c \geq 0$ is the linear premium rate. The
 $C_i$'s, $i=1,2,...$ are independent identically distributed (i.i.d)
random variables with distribution ${F(z)=F_C(z)}$ representing
nonnegative jumps arriving after independent exponentially distributed
times with mean $1/\lambda$, and $N_\lambda (t)$ denotes the associated Poisson
process counting the arrivals of claims on the interval $[0,t]$. Finally, $\s B(t),  \s >0$
is an independent Brownian perturbation.
 
 {\bf First passage theory}  concerns the first passage times above and below, and the hitting  time of a level $b$. For any process $X_t, t \geq 0$, these are defined by
\begin{equation}\la{fpt}
\begin{aligned}
\tb &= \tb^{X}=\inf\{t\geq 0: X_t> b\},\\
\ta&=\ta^X=\inf\{t\geq 0: X_t< a\},\\
T_{\{b\}}&=T_{\{b\}}^X =\inf\{t\geq 0: X_t=b\},
\end{aligned}
\end{equation}
with $\inf\emptyset=+\infty$. The upper script $X$ will be typically  omitted, as well as the signs $+,-$, when they are clear from the context.


Ruin happens when, for the first time, a jump takes $X_t$ below 0. Risk theory revolved initially around evaluating and minimizing the probability of ruin. As more and more tools were developed, and the need arose, insurance companies became interested in maximizing aspects of company value related to ruin. This lead to the study of optimal dividend policies.  As suggested by De Finetti in the 1950's \cite{deF} -- see also \cite{miller1961dividend} -- an interesting objective is that of maximizing the expected value of the sum of discounted future dividend payments until the time of ruin.

%\sec{De Finetti optimization of dividends  \la{s:div}}

The most important class of dividend policies is that of a constant barrier
at $b$, which modifies the surplus only when $X_t>b$, by a lump payment
bringing the surplus at $b$, and then keep it there by Skorokhod
reflection, until the next negative jump.  In financial terms, in the
absence of a Brownian component, this amounts to paying out all the income
while at $b$.  In case of Brownian perturbation, Skorokhod reflection means
keeping the process above the barrier by minimal capital injections
(whenever necessary), or below a barrier, by taking out dividends (if
necessary) \cite{sk62}.

In presence of the barrier at $b$, the de Finetti objective (the expected
value of the sum of discounted future dividend payments until ruin) has a
simple expression \cite{APP} in terms of the so called ``scale function''
$W$ introduced by \cite{Suprun,Ber}:
\begin{equation}
V^{b]}(x)=E^{b]}_x\le[\int_{[0,T_0^{b]}]}e^{-\q
   t}d \D_t^b \ri]= \bc \frac{W_\q( x)}{W_{\q}^ {\prime}(b)}, & x \leq
   b\\x-b + \frac{W_\q( b)}{W_{\q}^ {\prime}(b)}, & x>b\ec, \la{div}
\end{equation}
where $T_0^{b]}$ is the time of ruin, $q$ denotes the discount rate,
%$\D_t^b:=\le(\sup_{0 \leq s\leq t} X_{s}-b\ri)_+$
$\D_t^b$
the total local time at
$b$ before time $t$, and $E^{b]}$ the law of the process reflected from
above at $b$ and absorbed at $0$ and below.

\begin{comment}
\fn[4]{Formula \eqr{WLT}
reflects the representation $$V^{b]}(x)=E[ e^{-q \tb} ; \tb
< \tz] Eazb_b\le[\int_{[0,T_0^{b]}]}e^{-\q t}d \D_t\ri]=E[ e^{-q \tb}
; \tb < \tz]\; Eazb_b\le[ \D_{T_0^{b]} \wedge \kil_q}\ri],$$ and the fact
that the local time $\D_t$ at $b$ with reflection at $b$ is an
exponential \rv.}
\end{comment}

The scale function 
$W_{\q}(x):\R \to [0, \infty), \q \geq 0$ is defined on the positive
half-line by the Laplace transform
\be \label{WLT}
\H W_\q (s):=\int_0^\infty  \mathrm{e}^{-s x}  W_{\q}(x) d x = \frac {1} {\k(s)-\q} , \quad \for s > \Fq,
 \ee
where the ``symbol" $\k(s)$ (also called \cgf) is defined in \eqr{kf} in
Section~\ref{s:RT} where we provide the necessary background information,
and $\Fq$ is the unique nonnegative root of the \CL equation
\be \Fq:= \sup \{ s \geq 0: \k(s)
- \q= 0\}, \quad \q \geq 0. \label{Fq} \ee

The scale function $ W_{\q}(x)$ is continuous and increasing on
$[0,\I)$ \cite{Bingham}, \cite[Thm.~VII.8]{Ber}, \cite[Thm.~8.1]{Kyp}. It
may have \how many inflection points (such an example is depicted in
Figure \ref{f:pd}), and these play an important role in the optimization
of \divs \ \cite{APP,Schmidli,APP15}.  For convenience, $ W_{\q}(x)$ is
extended to be $0$ on $\R_-$. An important fact that will be exploited is
that the \LT \ of our function has a unique non-negative pole $\Fq$,
see \eqr{WLT} and \eqr{Fq}.


This paper aims at computing/approximating the scale function $W_{\q}(x)$,
using its moments. The techniques being used are classic: \Pd and Laguerre
expansions.  The order $(m,n)$ Pad\'e approximation of a function $g(x)$ is
a rational function in the form
\[
R(x)=\frac{a_0+a_1 x+a_2 x^2+...+a_m x^m}{1+b_1 x+b_2 x^2+...+b_n x^n}
\]
for which
$R(0)=g(0),R'(0)=g'(0),R''(0)=g''(0),...,R^{(m+n)}(0)=g^{(m+n)}(0)$.  In
the context of probability distributions, given a density function $f(x)$
and its \LT \ $\H f(s)$, the inverse Laplace transform of the order $(m,m)$
Pad\'e approximant of $\H f(s)$ provides a matrix exponential approximation
of $f(x)$ that matches the first $2m$ moments of $f(x)$ (including $m_0$).
In \cite{avram2011moments} this approach was used to approximate ruin probabilities.  In
this paper we develop the same approach to approximate the scale function
$W_{\q}(x)$ (Section~\ref{s:Pade}).  An extension of the above idea is the
so-called two-point Pad\'e approximation which allows to match not only the
moments of $W_{\q}(x)$ but also the behavior of the function at 0, i.e., to
match $W_{\q}(0)$, $W'_{\q}(0),...$ (Section~\ref{sec:low1}).
For more details of this extension see  \cite{ABH} where ruin
probabilities are approximated.


\begin{comment}
  \beR   Note that when $\q=0$, the scale function $W_0$ coincides, up to
  a \prop  \ct,  with the well-studied  \surp \ (just compare \eqr{WLT}
  with the famous \PK  \LT \ \eqr{PK0}).
  It appears thus that at the slight additional expense of inverting
  the  \LT \ \eqr{WLT} instead of  \eqr{PK0}, one may obtain the solution of numerous sophisticated control problems \cite{Kyp,AGV} (which are however similar
  in effort to obtaining ruin probabilities).
\eeR
\end{comment}

Let us draw attention now to several numeric challenges which were absent in the \rp \prob.
\BEN \im
{\bf Optimizing dividends} starts by optimizing the so called ``barrier function'' \be H_D(b):=\frac {1}{ W'_\q(b)}, \; b \geq 0, \la{GDeF}\ee
   and   the optimal dividend policy is  often simply a barrier strategy at its   maximum. This is the case in particular when the barrier function $H_D(b)$  is differentiable
with
\be H_D'(0) >0 Eq W''_\q(0) <0\ee
and has a unique local maximum $b^*>0 \Lra W''_\q(b^*)=0$; then this $b^*$ yields the optimal dividend policy, and
the optimal barrier function,
  \be \la{dbpr} V(x):= sup_{b \geq 0} V^{b]}(x)=V^{b^*]}(x),\ee
  turns out to be the largest concave minorant  of  $W_\q(x)$.\fn[4]{Even when barrier strategies do not  achieve the  optimum,  and  multi-band policies must be used instead, constructing the solution must start by
  determining the global maximum of the barrier function   \cite{AM05,Schmidli,APP15}.}

\im {\bf The challenge of multiple inflection points}.
  In the presence of several inflection points, however,  the optimal policy is multiband \cite{AM05,Schmidli,Loef,APP15}.
The first numerical examples of multiband policies  were produced in \cite{AM05,Loef}, with  Erlang claims $Erl_{2,1}$. However, it was shown in \cite{Loef} that multibands cannot occur when $W'_\q(x)$ is increasing after its  last global minimum  $b^*$ (i.e., when no local minima are allowed after the global minimum).


  \cite{Loef} further made the interesting  observation that for Erlang claims $ER_{2,1}$  (which are non-monotone), multiband policies may occur  for volatility \pars \ $\s$ smaller than a threshold value, but  barrier policies (with non-concave value function!) will occur when $\s$ is large  enough.

 Figure \ref{f:pd} displays the first derivative $W_\q'(x),$ for $\s^2/2 \in \{\frac{1}{2}, 1,\frac{3}{2},2\}$. The last two values yield barrier policies with non-concave value function, due to the presence of an inflection point in the interior of the interval $[0,b^*]$.
   \figu{pd-eps-converted-to}
   {Graphs of the Loeffen example for $\k(s)= \frac{\s^2 s^2}{2}+c \; s+\l
   \left(\frac{1}{(s+1)^2}-1\right), c=\frac{107 }{5}, \l=10, q=\frac{1}{10} $, $\s^2/2 \in \{ \color{blue} 1/2, \color{yellow}1, \color{green} 3/2, \color{red} 2 \color{black}\} $.}{0.6}




\EEN

 Below we will investigate whether our approximations are precise enough to
 yield reasonable approximations for $ W''_\q(0) $ and the root(s) of $
 W''_\q(\cdot)$.

{\bf Special features}. While our methods consist essentially of \Pd \
and \LTW \LT \ inversion, we found that exploiting the special features of
our problem is useful. These are:\BEN \im including known values of
$W_\q(0), W_\q'(0)$ (using thus two-point \Pds). \im shifting the
approximations around $\Fq$ specified in \eqr{Fq}, which transforms
$W_\q(x)$ into a \surp. As a consequence, we end up using a certain
judicious choice of the Laguerre exponential decay parameter \eqr{al},
which is usually left to be tuned by the user in the \LTW \
method \cite{weideman1999algorithms}.
\EEN


% -*- TeX:EN:UNIX; ispell-local-dictionary: "american"; TeX-master: "ratrisk"; -*-

{\bf Contents}. We briefly review  classical ruin theory in Section \ref{s:RT}. Pad\'e approximations are provided  in Section \ref{s:Pade}, where we also
    spell out  the simplest algorithm for the  computation of the \sf. % \ and of the optimal dividend barrier.
   %The de Finetti dividends \prob \ for the  spectrally negative \lev risk model is reviewed in section \ref{s:div}.
In Section~\ref{sec:low1}, we derive  low-order Pad\'e and two-point Pad\'e approximations of $W_\q(x)$, reminiscent
   of the  \deV approximation of the ruin probability. Some of these \app s  appeared already in \cite{gerber2008methods}, where however the Pad\'e method and the fact that they can be easily extended to higher orders is not mentioned.
  Section~\ref{s:two} offers our personal strategy for inverting \LTs \ of interest in probability, in the presence of uncertainty.
 Subsection~\ref{s:LTW} implements the Laguerre-Tricomi-Weeks \LT \ inversion  with a certain judicious choice of the exponential decay parameter \eqr{al}, which is believed to be new.  Section~\ref{s:mix} presents numeric experiments
   with \mix claims.
   Section~\ref{s:ht} presents experiments
   with Pareto claims; since these have heavy tail and, consequently, finitely many
    moments, we apply ``shifted'' Pad\'e approximation of the claim distribution.
   %for high order approximations, one is forced to make appeal to shifted Pad\'e approximation (SP), which is reviewed in Section~\ref{P}.
   Section~\ref{s:WH} includes a computer program required to obtain test cases with exact rational answers, using the \WH \ factorization; of course, this is quite convenient for the initial testing  of the precision of our algorithms.
   Finally, Section~\ref{s:Tricomi} reviews a more general version of the \LTW \ Laplace inversion method, which may be of interest for further experiments.
   %Section~\ref{s:Gam} presents numeric experiments
   %with the Gamma risk process.
\iffalse The subordinator risk model  perturbed by Brownian motion
 is introduced in section \ref{s:soph} -- see \eqr{eq:per} for  the two  formulas for ruin \pros (by creeping and by jump).
 \fi


\iffalse
Section~\ref{P} provides a probabilistic interpretation of shifted Pad\'e approximation (SP).  An illustration
of the superiority of SP on classic Pad\'e is provided in Section \ref{s:uni} on  a case with compact
support, the third example given in \cite{gzyl2013determination}
with claims of uniform law $\mL(C_i)=\mU_{[0,1]}$.  We have found that
Pad\'e and shifted Pad\'e approximations provide matrix exponential
approximations of the claim density that are not admissible on $[0,\infty)$
(i.e., the density is not always positive) but the $L_1$ errors of these
approximations are much lower than that of some approximations that impose
non-negativity. Furthermore, the resulting ruin probabilities are quite precise, and may be  admissible even when the density claim approximations aren't.


 In Section \ref{id9} we turn to heavy-tailed claims.
%and the programs used are offered at the web page \red{...}.
As documented, {SP approximation leads to very
accurate approximation of ruin probabilities for Pareto claim size when
compared to results obtained using the fixed Talbot algorithm
(FT) \cite{abate2004multi}, which is quasi-exact in this kind of problems,
see \cite{AAK}.
\fi

\section{A short review of classical ruin theory} \la{s:RT}


 The process defined in (\ref{CLp}) is  a particular example of  spectrally negative L\'evy processes,   with finite mean,  which are defined by assuming instead of \eqr{CLp} that $S_t$ is a  subordinator with $\s$-finite \Lm \ $\lm(dx)$ that integrates $x$, but having possibly infinite activity  near the origin $\lm(0,\I)=\I$ \cite{Ber} (for \eqr{CLp}, the \Lm \ is given by $\lm(dx)=\l F(dx) \Lra \lm(0,\I)=\l$).
 A spectrally negative L\'evy process is characterized by its
L\'evy-Khintchine/Laplace exponent/{cumulant generating function/symbol} $\k(s)$ defined by $E_0 \big[e^{s X(t)}\big]=e^{t \k(s)}$, with $\k(s)$ of the particular form
\be  \k(s)=  c s +   \int_0^\I
(e^{-s x}-1) \lm(d x)+\frac {\s^2 s^2}2 :=s \le(c -
\H{\ol{\lm}}(s)
+\fr{\s^2}2 s\ri).\la{kf}
\ee

Some concepts of %central
interest in classical risk theory are:
\BI
\im   { \em First passage times}
below and above a level $a$:  \begin{equation*} T_{a,-(+)} : = \inf\{t > 0: X(t) <(>)
a\}.\end{equation*}
%We will denote $\tz$ by $\t$.
\iffalse
\im The  {\em maximal aggregate loss} $L$
\beq \la{e:L} L_T:=\max_{0 \leq t < T} S(t)- ct, %-\s B(t)
L:=L_\I
\eeq
(the negative of the all-time infimum of the process
${X}$  started from $0$).
\fi
\im The first first passage quantity to be studied  historically was the  {\em eventual ruin probability}:
\be \la{rp0}\Rui(x):=P[\tz <\infty | X(0) = x].%= P[L >u].
\ee

In order that the eventual ruin probability  not be identically $1$, the  parameter
$$  p:=c -\l m_1 =\k'(0),  \text{ where } m_1=\int_0^\I z F(dz),$$
which is called  drift or profit rate,  must be assumed positive.

The \LT \ of the \rp
is explicit, given by the so called \PK formula, which states
 that the \LT \ of $\ovl \Rui(x)=1-
\Rui(x)$ is:
\be\label{PK0} \H \sRui (s)=\int_0^\I e^{-s x} \ovl \Rui(x)dx=\frac{c-\l m_1}{\k(s)}=\frac{\k'(0)}{\k(s)}.\ee
 \EI



The  roots with negative real part of the Cram\'er Lundberg equation \be \la{CLeq} \k(s)= \q, \q \geq 0\ee are  important, when such roots exist. They will be denoted by  $- \g_1, - \g_2 , \cdots, - \g_N, N \geq 0 $, and ordered by  their absolute
values  $| \g_1| \leq  |\g_2 | \leq ..., \leq | \g_N|$. $\g_1>0$ is  called the adjustment coefficient, and furnishes
the Cram\'er-Lundberg asymptotic approximation $$\Rui(u) \sim \fr{\k'(0)}{-\k'(-\g_1)} e^{-\g_1
u}.$$

\section{De Vylder approximation}
 An approximation of the ruin probability, in the absence of the Gaussian component,is given explicitely in this section by the \textit{De Vylder} approximation which is a Padé approximation of the \textit{Pollaczek-Khinchin} transform for the classical \textbf{Cramèr-Lundberg} model.\\
 In the case of exponentially distributed claim size, the main idea of De Vylder is based on replacing the risk reserve process {$X(t)$} by another risk process { $\tilde{X}(t)$} such that for some $n\geq 1$ the moments up to order $n$ of certain characteristics of {$ X(t)$} and {$\tilde{X}(t)$} coincide. \\
 \begin{equation*}
 X(t)-u = pt -\ \sum_{i=1}^{N_{t}} (C_{i} -\ EC_{i}) = \theta \lambda m_{1}t - \ \sum_{i=1}^{N_{t}} (C_{i} -\ EC_{i})
 \end{equation*} \\
 Where $ p:= c- \ \lambda m_{1} =\lambda m_{1} \theta \ > 0$ is the profit rate. \\

 Furthermore, {$\tilde{X}(t)$} is chosen in such a way that the ruin probability $\tilde{\Rui}(u)$ of {$\tilde{U}(t)$} is easier to determine than $\Rui (u)$. \\ \\
Let us designate by $\mathcal{K}_{k}(Y)$ the k-order cumulant of a random variable $Y$ , i.e. the  coefficients of the Taylor expansion:
\begin{equation*}
log( \mathbb{E}e^{u Y}) =\sum_{k=0}^{\infty} \mathcal{K}_{k}(Y)\frac{u^{k}}{k!}
\end{equation*}
\subsection{Proposition:}
$a)$The k-order cumulant of a compound Poisson process $L(t)$ is of the form: \\
\begin{equation*}
\mathcal{K}_{k}(L(t)) =\int_{0}^{t} \mathcal{K}_{k}[X(ds)]ds = \lambda t  \int x^{k}f(x) dx, \ \forall k\geq 1.
\end{equation*} \\
$b)$
\begin{equation*}
\mathcal{K}_{1}(X(t))= pt , \ \mathcal{K}_{2}(X(t)) =t(\sigma^{2} +\lambda \int x^{k}f(x) dx) , \ \mathcal{K}_{k}(X(t))= (-1)^{k}\lambda t \int x^{k} f(x) dx , \ \forall k\geq 3.
\end{equation*}
\subsection{DeVylder Approximation:}
The idea of the \textit{De Vylder} approximation is to replace $X(t)$ by $\tilde{X}(t)$, where $ \tilde{U}(t)$ has exponentially distributed claim size and : \\
\begin{equation}
 \mathcal{K}_{k}(X(t)) = \mathcal{K}_{k}(\tilde{X}(t))
\end{equation}. \\
Let $\tilde{\lambda}, \ \tilde{p}, \ \tilde{m}$ et $\tilde{\theta}$ the new parameters of $\tilde{U}(t)$, representing respectively the average rate of arrival claim, the profit rate, the loading security coefficient, and the average claim reimbursement amounts. \\
From \textit{proposition 4.2} ,this implies that \textcolor{red}{(10)} holds if :
\begin{equation}
\left\lbrace
\begin{aligned}
p= \theta \lambda m_{1} = \tilde{p} =\tilde{\theta}\tilde{\lambda} \tilde{m}_{1} \\
\lambda m_{2} = 2\tilde{\lambda}\tilde{m}^{2} \\
\lambda m_{3} = 6\tilde{m}^{3}
\end{aligned}
\right.
\end{equation}
\newpage
A computation leads to : \\
$$\lambda = \frac{2 \tilde{\lambda}\tilde{m}^{2}}{m_{2}}$$
 $$\tilde{\lambda} =\frac{\lambda m_{3}}{6 \tilde{m}^{3}} =\frac{\lambda m_{3}}{6 (\frac{m_{3}}{3 m_{2}})^{3}} = \frac{9 m_{2}^{3}}{2 m_{3}^{2}}\lambda $$ \\
 And
$$\tilde{\theta}= \frac{\lambda m_{1}}{\tilde{\lambda}\tilde{m}}\theta =\frac{2 m_{1}m_{3}}{3m_{2}^{2}}\theta$$ \\
Now we use the first $2n$ ruin moments $ \lambda_{1}=\frac{\tilde{m}_{1}}{\theta}, \ \lambda_{2}= \frac{\tilde{m_{2}}}{2 \theta} + \ (\frac{\tilde{m}^{2}}{\theta})^{2}, ..., \lambda_{2n-1}.$ \\
For $n=1$, one may set up a system for the Padé $(0,1)$ approximation in $0$ , and from the Laplace transform of ruin probability in \textcolor{red}{(4)}, we obtain:
\\
\begin{equation*}
\begin{split}
\widehat{\Rui}_{DV}(s)= \frac{1}{s} - \ \frac{p}{s(p+ \lambda m_{2}s/2 -\lambda m_{3}s^{2/6}+...)} \\ \\
= \frac{\lambda m_{2}/2 - \lambda m_{3}s/6+...}{p+  \lambda m_{2}s/2 -\lambda m_{3}s^{2/6}+...}
\approx \frac{a}{s+b} \\ \\
\Leftrightarrow as( p+  \lambda m_{2}s/2 -\lambda m_{3}s^{2/6}+...) \\ \\
\approx (s+b)( \lambda m_{2}s/2 - \lambda m_{3}s^{2}/6+...) \\ \\
\Leftrightarrow ap= b \lambda m_{2}/2 , \ a m_{2}/2= m_{2}/2 - \ b m_{3}/6 \\ \\
\Leftrightarrow
a= \frac{3 \lambda m_{2}^{2}}{3 \lambda m_{2}^{2} + 2p m_{3}} , \ b= \frac{6 p m_{2}}{3 \lambda m_{2}^{2}+ 2 p m_{3}}.
\end{split}
\end{equation*} \\
This leads to the \textbf{DeVylder}'s famous approximation.
\begin{equation}
\Rui_{DV}(u) \approx a e^{-bu}
\end{equation}
\newpage

Hence
\begin{equation}
\Rui_{DV}(u) = \frac{1}{1+ \tilde{\theta}} \ exp \left( - \frac{u \tilde{\theta} \mu}{1+ \tilde{\theta}} \right)
\end{equation}
Where $ \mu =\frac{3 m_{2}}{m_{3}}, \ \tilde{\theta}= \frac{2 p m_{3}}{3 \lambda m_{2}^{2}} = \ \frac{2 m _{1} m_{3}}{3 m_{2}^{2}} \theta$ represent the exponential reimbursement rate and the relative safety factor in the associated process $\tilde{X}(t)$, respectively.




\section{De Vylder approximation}

Computing the ruin probability using the \PK formula is not always straightforward, as the problem of finding an inverse Laplace transform of a given expression is not trivial.

A technique in approximating functions whose Laplace transforms are known involves approximating the Laplace transform by a certain rational expression. The rational expression can then be broken down into partial fractions whose inverse Laplace transform is known.

From the idea that
\begin{align*}
\H{\ol{F}}(s) &= \int_{0}^{\infty} \ol{F}(x) e^{-sx} ~dx = \int_{0}^{\infty} \ol{F}(x) \sum_{k=0}^{\infty} \frac{(-sx)^k}{k!} ~dx \\
& = \sum_{k=0}^{\infty} \frac{(-s)^k}{k!} \int_{0}^{\infty} \ol{F}(x) x^k ~dx = \sum_{k=0}^{\infty} \frac{(-s)^k}{k!} m_{k+1},
\end{align*}
we get
\begin{align*}
\k(s) &= cs + \frac{\sigma^2 s^2}{2} - \lambda s \H{\ol{F}}(s) = cs + \frac{\sigma^2 s^2}{2} - \lambda s \sum_{k=0}^{\infty} \frac{(-s)^k}{k!} m_{k+1}\\
&= cs + \frac{\sigma^2 s^2}{2} - \lambda s \left( m_1 - s m_2 +  \frac{s^2}{2} m_3 + \cdots \right)
\end{align*}

Hence if we approximate $\H{\ol{\Psi}}(s)$ by a rational expression, say $\frac{a}{s+b}$ (where $a$ and $b$ are coefficients to be solved for), we have
\begin{align*}
& \H {\Psi}(s) = \frac{1}{s} - \frac{\kappa^{\prime}(0)}{\kappa (s)} = \frac{\kappa (s) - s \kappa^{\prime} (0)}{s \kappa (s)}  \approx \frac{a}{s+b} \\
\Rightarrow & as \kappa (s) \approx (s+b) \left( \kappa (s) - s \kappa^{\prime} (0) \right) = (s+b) \left( \kappa (s) - s (c- \lambda m_1) \right)
\end{align*}

Simplifying a term at the right side of the equation leads to 
\begin{align*}
\kappa (s) - s (c- \lambda m_1) & = cs + \frac{\sigma^2 s^2}{2} - \lambda s \left( m_1 - s m_2 +  \frac{s^2}{2} m_3 + \cdots \right) -cs + s\lambda m_1\\
& = \frac{\sigma^2 s^2}{2} + \lambda s^2 m_2 - \lambda \frac{s^3}{2} m_3 + \cdots
\end{align*}
which implies
\[
as left( cs + \frac{\sigma^2 s^2}{2} - \lambda s \left( m_1 - s m_2 + \cdots \right) right) \approx (s+b) \left( \frac{\sigma^2 s^2}{2} + \lambda s^2 m_2 - \lambda \frac{s^3}{2} m_3 + \cdots \right)
\]
\[
\Rightarrow a \left( c + \frac{\sigma^2 s}{2} - \lambda \left( m_1 - s m_2 + \cdots \right) \right) \approx (s+b) \left( \frac{\sigma^2}{2} + \lambda m_2 - \lambda \frac{s}{2} m_3 + \cdots \right).
\]

One can solve for the coefficients $a$ and $b$ by truncating the sums and comparing the coefficients of the first few powers of $s$, which gives us the system
\[
a
\]
\section{Which exponential   approximation? \la{s:DV}}
\ssec{Some de Vylder-type approximations for the \rp\ \la{s:DVr} }
In the simplest case of \expoj\ and $\s=0$, the  formula for the \rp\ is
\begin{equation} \la{Ruiex}
\Rui(x) = P_x[ \exists t \geq 0: X_t < 0]= \frac{1}{1+  {\th}} \ exp \left( - \frac{x  {\th} m_1^{-1}}{1+  {\th}} \right),
\end{equation}
where $\th=\fr{c -\l m_1}{\l m_1}$ is the \lc.
This formula may serve  as the {\bf simplest approximation} for \procs\ with general claims, simply by plugging the mean of the claims.


{\bf The Renyi and De Vylder's approximations}.
More sophisticated is the Renyi exponential
approximation
\begin{equation*}
\Rui_R(x) =  \frac{1}{1+  {\th}} \ exp \left( - \frac{x  {\th} m_R^{-1}}{1+  {\th}} \right), m_R=\frac{m_{2}}{2m_{1}};
\end{equation*}

This formula can  be obtained as a \tPd\   of the \LT, which conserves also the value $\Rui(0)=(1+\th)^{-1}$ \cite{AP14}. It may be also viewed heuristically as an exponential
approximation of the {\bf excess/severity density} $$f_e(x)=\fr{\ovl F(x)}{m_1}=\fr{1- F(x)}{m_1}.$$ Heuristically, it makes more sense to make $f_e(x)$ exponential instead of the original density $f(x)$, since $f_e(x)$ is a monotone function (and also an important component of the \PK formula for the \LT\ $\H \Rui(s)=\int_0^\I e^{- s x} F( dx)$) -- see \cite{ramsay1992practical,AP14}.




More moments may be put to work in the \deV
\begin{equation}
\Rui_{DV}(x) = \frac{1}{1+ \T {\th}} \ exp \left( - \frac{x \T {\th} \T m^{-1}}{1+ \T {\th}} \right), \; \T {m}=\frac{m_{3}}{3m_{2}}=\H m_{3}, \;  \T \th =\frac{2 m_{1}m_{3}}{3m_{2}^{2}}\th=\frac{\H m_{3}}{\H m_{2}}\th.
\end{equation}
Interestingly, the result may be expressed in terms of the so-called "normalized moments" \be {\H m_i}=\fr{m_i}{i \; m_{i-1}}\ee  introduced in \cite{bobbio2005matching}.

The \deV\ parameters above may be obtained either from  \BEN \im a \Pd\ of the \LT\ of the \rp\ \cite{avram2011moments}, or \im by equating the first  cumulants of our process to those of a \proc\ with  exponentially distributed claim sizes of mean $\T m$,  and  modified $\l, c$  \cite{de1978practical};  (however   $p=c -\l m_1$ must be conserved, since this is the first cumulant). This results -- see \eqr{ap3} -- in the modified \paras
$$\T m =\H m_3, \;  \T {\lambda}
= \frac{9 m_{2}^{3}}{2 m_{3}^{2}}\lambda=\fr{ m_2/2} { \H m_3^2} \l, \; \;  \T c= c -\l m_1+ \T \l \T m= c -\l m_1+  \l \frac{ m_{2}^{2}}{2 m_{3}}$$
where we gave also expressions involving the less standard normalized moments $\H m_k$ of \cite{bobbio2005matching}.
\EEN

The first derivation via Pad\'e shows that higher order approximations may be easily obtained as well (but they might not be admissible, due to negative values).

The second derivation of  De Vylder is a {\bf process approximation} (i.e., independent of the problem considered); as such, it may  be  applied to other functionals of interest  besides \rps ($W_q(x)$, dividend barriers, etc), simply by plugging the modified \para s in the exact formula for the \rp\ of the simpler process.






\iffalse
\section{Can we minimize the \rp, by proportional \rei? \la{s:DVr} }

Let us consider first the mean premium principle, with $\th_R \geq \th$ and retention level (potentially dependent on $x$). As \wk, we may forget about the reinsurer, and say instead that we modify the premium to
and the \lc\ to
\be \la{Tc} \T c=\l m_1 \pr{1+ \th - (1-a) \pr{ \th_R +1}}=\l m_1 \pr{1+ \T \th},\ee
\be \la{pos} \T \th=\th - (1-a) \pr{ \th_R +1},\ee
with $a \geq \fr {1 +\th_R-\th}{1+ \th_R } $, to ensure positivity in \eqr{pos}.
 \Fr we must also multiply the mean by $a$. %the level of the naive approximation

 The \rp\ has one critical point with $a$ satisfying  a quadratic equation (but, taking $a$ at the critical point  yields negative $\T \th$). A careful analysis reveals that the optimum is at $a=1$?.

 This continues to be the case for the mean-variance premium principle (Shihao?).
 The optimization becomes non-trivial for excess of loss \rei?.
\fi

The Laplace transform of $ \psi(x)$ and $\overline{\psi}(u)$ are respectively given by the so called \textbf{Pollaczek-Khinchine} :  \\
 \begin{equation}
 \begin{split}
 \widehat{\psi}(s)=\int_{0}^{\infty}e^{-sx}\psi(x)dx =\frac{1}{s}- \frac{1-\rho}{s-\frac{\lambda}{c}(1-\hat{f}(s))}=\frac{1}{s}\left( 1-\frac{1-\rho}{1-\rho \hat{f}_{e}(s)}\right).
 = \rho\frac{\widehat{\overline{F}}_{e}(s)}{1-\rho\hat{f}_{e}(s)} \\ \\
 \Leftrightarrow \widehat{\overline{\psi}}(s)= \int_{0}^{\infty}e^{-sx}\overline{\psi}(x)dx =\frac{c(1-\rho)}{k(s)}=\frac{1-\rho}{s(1-\rho\hat{f}_{e}(s))}
  \end{split}
 \end{equation} \\
 Where $\hat{f}(s)=\int_{0}^{\infty}e^{-sx}dF(x)$, $\overline{F}(x) =1-F(x)$ \\
  $f_{e}(x)= \overline{F}(x)/m_{1}$:\textit{ which denotes the stationary excess density of the claims; ($\int_{0}^{\infty} \overline{F}(x)dx = m_{1})$} \\
  $\hat{f}_{e}(s)= \int_{0}^{\infty}e^{-sx}f_{e}(x)dx =(1-\hat{f}(s))/(m_{1}s)$,$ \widehat{\overline{F}}_{e}(s)=(1-\hat{f}_{e}(s))/s$. \\
 \subsection*{Proof:}
 Using the Laplace transform of the ruin probability that satisfy the integro-differential equation of \textcolor{blue}{[IfM3, eq.(13.9)]} which gives: \\
 \begin{equation}
 \hat{\psi}(s)= \frac{c\psi(0)- \lambda \widehat{\overline{F}}(s)}{cs(1-\lambda \widehat{\overline{F}}(s))}
 \end{equation} \\
 Now by substituting $s=0$ in the denominator we can clearly see that the denominator explodes into zero which means that we have a singularity at this point. We assume that our probability of ruin declines fast enough to have a finite integral .Then in order to have $\widehat{\psi}(0)=\int_{0}^{\infty}\psi(x)dx < \infty$ it is necessary to check that the numerator admits also a singularity in zero. \\
 To achieve this we'll need a results that's easy to demonstrate based on integration by parts, we obtain the Laplace transformation of the survival function in terms of the density transformation mentioned above, we get: \\
 \begin{equation}
 \widehat{\overline{F}}(s)= \frac{1-\hat{f}(s)}{s} , \\ \widehat{\overline{F}}(0)= \int_{0}^{\infty}\overline{F}(x)dx = m_{1}
 \end{equation}\\
 As the denominator becomes $0$ for $s=0$ and $\widehat{\psi}(0)<\infty$ then: \\
 \begin{equation}
 \psi(0) =\frac{\lambda \int_{0}^{\infty}\overline{F}(x)dx}{c} =\frac{\lambda m_{1}}{c} := \rho.
 \end{equation}\\
 For the survival probability, we use $ \overline{\psi}(x)= 1- \psi (x)$. Thus we obtain
\begin{equation}
  \widehat{\overline{\psi}}(s) =\frac{1}{s} - \widehat{\psi}(s). , \\ \overline{\psi}(0)= 1- \frac{\lambda m_{1}}{c} = 1-\rho.
  \end{equation} \\
 Hence the result.

 \subsection{Corollary:}
 The ruin probability $\psi(u)$ is  $1$ for all u when $\rho \geq 1$ , and $< 1$ for all u when $ \rho< 1$.
 \newpage

 \section{The Pollaczek-Khinchine formula}
We shall here exploit the decomposition of the claims process as sum of ladder heights. Since our model is a \textit{compound Poisson process}, then the ladder heights are i.i.d. \\
Let us recall the Laplace transform defined in \textcolor{red}{(4)}.\\
 We have $ \frac{1}{1-\rho \hat{f}_{e}(s)}$ is expandable in a geometric series, and by inverting the Laplace transform we get:
\begin{equation}
\overline{\psi}(u)= \sum_{n=0}^{\infty} (1-\rho) \  \rho^{n} \ F_{e}(u)^{\star n} \ \Leftrightarrow \psi (u)= \sum_{n=1}^{\infty} (1-\rho) \  \rho^{n} \  \overline{F}_{e}(u)^{\star n}
\end{equation} \\
Reciprocally, from \textcolor{red}{(9)}, we calculate its Laplace transform, the convolution becomes product, and easily we find \textcolor{red}{(4)}.\\
To intepret this formula, we would need the following definition.

\subsection{Definition:}
The variable $ Y= X_{\tau}\mid \left\{ \tau <\infty ,  \ X_{0}= u\right\} >0$  is the severity of ruin. \\ \\
Let $S_{\tau_{-}}$ denotes the first (weak) descending ladder heights past u (defined on $\left\{ \tau_{-}< \infty \right\}$ only).
\subsection{Geometric Interpretation:}
Let's take the graph of \textbf{Cramèr-Lundberg} model with initial capital u.\\
Let's assume that on our kth descending ladder heights ruin occurs. \\  The ruin severity density is denoted by $f_{e}$ as defined above. \\
We try to decompose the initial reserve of the Cramèr-Lundberg model as well as the severity of ruin as the sum of the kth ladder heights. \\
We will call a walk without jumps strictly greater than 1 up/down continuous up/down. So when we take a walk that goes up, the walk will have a finite number of ladder heights down, so the number of ladder heights down is finite, we will denote it by \textbf{N}. We assume that \textbf{N} is a geometric varible with parameter $\rho$.
\newpage
Thus, for $N=0$
\begin{equation*}
\overline{\psi}(u)= 1- \rho.
\end{equation*}
 and for $N=1$
 \begin{equation*}
 \overline{\psi}(u)= (1-\rho) \ \rho \ \mathbb{P}( S_{\tau_{-}}< \ u)
 \end{equation*} \\
 Further, The law of the ladder heights is the law of equilibrium $ F_{e}$. \\
 Therefore, \\
 \begin{equation*}
 \overline{\psi}(u) = (1-\rho) \ \rho^{n} \ F_{e}(u)^{\star u}
 \end{equation*} \\
 So the ruin occurs if and only if the total sum of the heights exceeds u.
 
\small
\bibliographystyle{amsalpha}
\bibliography{Pare36}


\end{document}

